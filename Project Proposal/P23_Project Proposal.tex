\documentclass{article}

% if you need to pass options to natbib, use, e.g.:
%     \PassOptionsToPackage{numbers, compress}{natbib}
% before loading neurips_2025

% The authors should use one of these tracks.
% Before accepting by the NeurIPS conference, select one of the options below.
% 0. "default" for submission
 \usepackage[final]{neurips_2025}
% the "default" option is equal to the "main" option, which is used for the Main Track with double-blind reviewing.
% 1. "main" option is used for the Main Track
%  \usepackage[main]{neurips_2025}
% 2. "position" option is used for the Position Paper Track
%  \usepackage[position]{neurips_2025}
% 3. "dandb" option is used for the Datasets & Benchmarks Track
 % \usepackage[dandb]{neurips_2025}
% 4. "creativeai" option is used for the Creative AI Track
%  \usepackage[creativeai]{neurips_2025}
% 5. "sglblindworkshop" option is used for the Workshop with single-blind reviewing
 % \usepackage[sglblindworkshop]{neurips_2025}
% 6. "dblblindworkshop" option is used for the Workshop with double-blind reviewing
%  \usepackage[dblblindworkshop]{neurips_2025}

% After being accepted, the authors should add "final" behind the track to compile a camera-ready version.
% 1. Main Track
 % \usepackage[main, final]{neurips_2025}
% 2. Position Paper Track
%  \usepackage[position, final]{neurips_2025}
% 3. Datasets & Benchmarks Track
 % \usepackage[dandb, final]{neurips_2025}
% 4. Creative AI Track
%  \usepackage[creativeai, final]{neurips_2025}
% 5. Workshop with single-blind reviewing
%  \usepackage[sglblindworkshop, final]{neurips_2025}
% 6. Workshop with double-blind reviewing
%  \usepackage[dblblindworkshop, final]{neurips_2025}
% Note. For the workshop paper template, both \title{} and \workshoptitle{} are required, with the former indicating the paper title shown in the title and the latter indicating the workshop title displayed in the footnote.
% For workshops (5., 6.), the authors should add the name of the workshop, "\workshoptitle" command is used to set the workshop title.
% \workshoptitle{WORKSHOP TITLE}

% "preprint" option is used for arXiv or other preprint submissions
 % \usepackage[preprint]{neurips_2025}

% to avoid loading the natbib package, add option nonatbib:
%    \usepackage[nonatbib]{neurips_2025}

\usepackage[utf8]{inputenc} % allow utf-8 input
\usepackage[T1]{fontenc}    % use 8-bit T1 fonts
\usepackage{hyperref}       % hyperlinks
\usepackage{url}            % simple URL typesetting
\usepackage{booktabs}       % professional-quality tables
\usepackage{amsfonts}       % blackboard math symbols
\usepackage{nicefrac}       % compact symbols for 1/2, etc.
\usepackage{microtype}      % microtypography
\usepackage{xcolor}         % colors

% Note. For the workshop paper template, both \title{} and \workshoptitle{} are required, with the former indicating the paper title shown in the title and the latter indicating the workshop title displayed in the footnote. 
\title{Semantic Mapping of FWH Building Data}


% The \author macro works with any number of authors. There are two commands
% used to separate the names and addresses of multiple authors: \And and \AND.
%
% Using \And between authors leaves it to LaTeX to determine where to break the
% lines. Using \AND forces a line break at that point. So, if LaTeX puts 3 of 4
% authors names on the first line, and the last on the second line, try using
% \AND instead of \And before the third author name.


\author{%
  P23: Yujin Kim, ykim68 (https://github.com/ykim68ncstate/ncsu-engr-ALDA-F25-Project-P23)\\
  % examples of more authors
  % \And
  % Coauthor \\
  % Affiliation \\
  % Address \\
  % \texttt{email} \\
  % \AND
  % Coauthor \\
  % Affiliation \\
  % Address \\
  % \texttt{email} \\
  % \And
  % Coauthor \\
  % Affiliation \\
  % Address \\
  % \texttt{email} \\
  % \And
  % Coauthor \\
  % Affiliation \\
  % Address \\
  % \texttt{email} \\
}


\begin{document}


\maketitle

\section{Data Set}

Our project will use HVAC operation data from the Fitts-Woolard Hall building on Centennial Campus. It contains one year of logsfrom more than 15sensors and equipment points, collected at 15-minute intervals.


\section{Project Idea}

Our Project goal is to build an auto-labeling model that maps building operation data into a standard format. In Fitts-Woolard Hall, sensor names are inconsistent, so machines cannot understand their meaning (e.g., "AHU1\_Temp\_{S1}" is an air handling unit sensor). This inconsistency makes it hard to use building data with other tecnologies or platforms.

We will implement and compare the performance of machine learning models for mapping building points to the Brick Ontology. The comparison will focus on how accurately and consistently each approach can classify diverse building data points.


\section{Software to Write}

We will write a Python tool to preprocess building data, extract features, and implement an XGBoost-based auto-labeling model that outputs Brick-compatible formats. We will evaluate the model with cross-validation using metrics such as macro F1-socre, precision, and recall.


\section{Relevant Papers}

\medskip

{
\small


[1] Xu, M., Wen, Z., Xiang, X., Fu, W., \& Whang, B. (2025) Winning Brick by Brick with Daily Slices: A 94-Task Unified XGBoost Solution for Brick Schema Classification. In {\it WWW Companion ’25}, April 28–May 2, 2025, Sydney, NSW, Australia.

[2] Zhao, H., Macken, J., Dinendra, L., He, Y., \& Yang, R. (2025) Hierarchical Multi-Label Classification of Building Management System Time-Series Data. In {\it WWW Companion ’25}, April 28–May 2, 2025, Sydney, NSW, Australia.

[3] Chan, J.H. (2025) BrickMIR: A Minimal, Imbalance-tuned, and Ratio-based Framework for Brick Metadata Classification. In {\it The WebConf ’25}, April 28–May 2, 2025, ICC Sydney, Sydney, Australia.
}


\section{Division of Work}

\paragraph{Yujin Kim.} Research relevant papers and design the XGBoost-based auto-labeling model. Implement data prepocessing, training, evaluation, and final report preparation.


\section{Midterm Milestone}

By the Midterm checkpoint, we will implement data preprocessing and feature extraction for the FWH dataset, and ddevelop a prototype XGBoost model for auto-labeling building points.



\end{document}